\documentclass[12pt]{article}
\usepackage[margin=1in]{geometry}
\usepackage{amsmath, amssymb}
\usepackage{graphicx}
\usepackage{float}
\usepackage{hyperref}
\usepackage{caption}
\usepackage{subcaption}

\title{Dynamic Programming for DNA Sequence Alignment}
\author{Xinyu Cheng, Zhihao Wang, Xueni Tang, Jieni Tang}
\date{August 2025}

\begin{document}

\maketitle

\tableofcontents
\newpage

\section{Introduction and Motivation (by Xinyu Cheng)}
\subsection{Rationale Behind DNA Sequence Alignment}
% Your content here

\subsection{Why Dynamic Programming?}
% Your content here

\subsection{Project Structure Overview}
% Describe layout of report and group contributions

\section{Mathematical Foundations (by Zhihao Wang)}
\subsection{Problem Definitions}
% Define Global and Local Alignment formally

\subsection{Dynamic Programming Principles}
% Discuss optimal substructure and overlapping subproblems

\subsection{Recurrence Relations}
% Present recurrence relations for global and local alignment

\subsection{Initialization and Traceback}
% Add notes on matrix setup and traceback strategies

\section{Implementation (by Xueni Tang)}
\subsection{Global Alignment Algorithm}
% Describe algorithm and code logic

\subsection{Local Alignment Algorithm}
% Describe algorithm and code logic

\subsection{Traceback and Output}
% Reconstructing alignments

\section{Runtime Analysis and Empirical Results (by Jieni Tang)}
\subsection{Theoretical Time and Space Complexity}
% O(mn) analysis and any optimizations

\subsection{Empirical Evaluation Setup}
% Datasets and testing setup

\subsection{Runtime Comparisons}
% Include a table or chart comparing runtimes

\section{Interpretation of Results and Practical Pitfalls (by Xinyu Cheng)}
\subsection{Issues and Limitations}
% Scoring ties, memory, ambiguous bases

\subsection{Trade-offs in Design}
% Accuracy vs performance

\section{Future Extensions (by Xinyu Cheng)}
% Ideas like affine gaps, banded alignment, GPU parallelism

\section{Conclusion (by Zhihao Wang)}
% Summarize what was learned and its broader implications

\newpage
\begin{thebibliography}{9}

\bibitem{nw}
Needleman, Saul B., and Christian D. Wunsch. "A general method applicable to the search for similarities in the amino acid sequence of two proteins." \textit{Journal of molecular biology} 48.3 (1970): 443-453.

\bibitem{sw}
Smith, Temple F., and Michael S. Waterman. "Identification of common molecular subsequences." \textit{Journal of molecular biology} 147.1 (1981): 195-197.

% Add more references here as needed

\end{thebibliography}

\end{document}
